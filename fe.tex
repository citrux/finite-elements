\documentclass[a4paper,12pt]{article}
\usepackage[russian]{babel}
\usepackage[utf8]{inputenc}
\usepackage[sumlimits, intlimits]{amsmath}
\begin{document}
	\section{Метод конечных элементов. Уравнение теплопроводности}
	Рассмотрим простейшую задачу теплопроводности:
	\[
		\frac{\partial u}{\partial t} = \frac{\partial^2 u}{\partial x^2},
		\quad x \in [0, 1],
	\]
	\begin{align*}
		& u(0, t) = u(1, t) = 0,\\
		& u(x, 0) = \sin \pi x.
	\end{align*}

	Она имеет точное решение:
	\[
		u(x, t) = e^{-\pi^2 t}\sin\pi x.
	\]

	Решим её численно с помощью метода конечных элементов в его проекционной
	формулировке. Для начала немного теории:
	\begin{enumerate}
		\item разложим искомую функцию по некоторому конечному функциональному
			  базису \( \{\psi_k\} \), разделив временную и пространственную
			  координаты:
			  \[ u(x, t) \approx \sum_{k=0}^n c_k(t)\psi_k(x); \]
		\item подставим это представление в уравнение и определим невязку:
			  \[
			  	R = \left(\frac{\partial}{\partial t} -
			  		\frac{\partial^2}{\partial x^2}\right)
			  		\sum_{k=0}^n c_k(t)\psi_k(x) =
			  		\sum_{k=0}^n
			  		\left(c_k(t)'\psi_k(x) - c_k(t)\psi_k(x)''\right);
			  \]
		\item минимизация невязки, а вместе с ней и наилучшее приближение
			  решения, достигается при ортогональности невязки базису:
			  \[
			  	\int_0^1 R\cdot\psi_m(x) dx = 0,\quad m = \overline{0, n};
			  \]
	\end{enumerate}

	Отсюда получаем систему дифференциальных уравнений:
	\[
		a_{mk}c_k(t)' = b_{mk}c_k(t),
	\]
	где
	\[
		a_{mk} = \int_0^1 \psi_m(x) \psi_k(x) \,dx,\quad
		b_{mk} = \int_0^1 \psi_m(x) \psi_k''(x) \,dx,
	\]
	которые зависят уже только от выбранного базиса.

	\section{Выбор базиса}
	\subsection{B-сплайны}
		Разобьём \( [0,1] \) на \( n \) равных частей и рассмотрим в качестве
		базисных функций \( B_1 \)-сплайны:
		\begin{align*}
			\psi_0 & = \begin{cases}
				\cfrac{x-x1}{x_0-x_1},\quad \text{если } x\in[x_0, x_1),\\
				0,\quad \text{иначе},
			\end{cases}\\
			\psi_k & = \begin{cases}
				\cfrac{x-x_{k-1}}{x_k-x_{k-1}},\quad \text{если } x\in[x_{k-1}, x_k),\\
				\cfrac{x-x_{k+1}}{x_k-x_{k+1}},\quad \text{если } x\in[x_k, x_{k+1}),\\
				0,\quad \text{иначе},
			\end{cases}\\
			\psi_n & = \begin{cases}
				\cfrac{x-x_{n-1}}{x_n-x_{n-1}},\quad \text{если } x\in[x_{n-1}, x_n),\\
				0,\quad \text{иначе}.
			\end{cases}
		\end{align*}

		Матричные элементы имеют вид:
		\[
			a_{mk} = \begin{cases}
				h/3, \text{ если } m=k=0, m=k=n,\\
				2h/3, \text{ если } m=k,\\
				h/6, \text{ если } m=k \pm 1,\\
				0,\quad \text{иначе},
			\end{cases},
			\quad
			b_{mk} = \begin{cases}
				0, \text{ если } m=k=0, m=k=n,\\
				-2/h, \text{ если } m=k,\\
				1/h, \text{ если } m=k \pm 1,\\
				0,\quad \text{иначе},
			\end{cases}
		\]

		Так как у этих сплайнов проблемы со второй производной, то для
		вычисления \( b_{mk} \) использовалось интегрирование по частям.

		Таким образом, матрица \( a_{mk} \) -- трёхдиагональная, а для
		нахождения \( A^{-1}B \) можно воспользоваться методом прогонки.
\end{document}